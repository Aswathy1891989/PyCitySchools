
% Default to the notebook output style

    


% Inherit from the specified cell style.




    
\documentclass[11pt]{article}

    
    
    \usepackage[T1]{fontenc}
    % Nicer default font (+ math font) than Computer Modern for most use cases
    \usepackage{mathpazo}

    % Basic figure setup, for now with no caption control since it's done
    % automatically by Pandoc (which extracts ![](path) syntax from Markdown).
    \usepackage{graphicx}
    % We will generate all images so they have a width \maxwidth. This means
    % that they will get their normal width if they fit onto the page, but
    % are scaled down if they would overflow the margins.
    \makeatletter
    \def\maxwidth{\ifdim\Gin@nat@width>\linewidth\linewidth
    \else\Gin@nat@width\fi}
    \makeatother
    \let\Oldincludegraphics\includegraphics
    % Set max figure width to be 80% of text width, for now hardcoded.
    \renewcommand{\includegraphics}[1]{\Oldincludegraphics[width=.8\maxwidth]{#1}}
    % Ensure that by default, figures have no caption (until we provide a
    % proper Figure object with a Caption API and a way to capture that
    % in the conversion process - todo).
    \usepackage{caption}
    \DeclareCaptionLabelFormat{nolabel}{}
    \captionsetup{labelformat=nolabel}

    \usepackage{adjustbox} % Used to constrain images to a maximum size 
    \usepackage{xcolor} % Allow colors to be defined
    \usepackage{enumerate} % Needed for markdown enumerations to work
    \usepackage{geometry} % Used to adjust the document margins
    \usepackage{amsmath} % Equations
    \usepackage{amssymb} % Equations
    \usepackage{textcomp} % defines textquotesingle
    % Hack from http://tex.stackexchange.com/a/47451/13684:
    \AtBeginDocument{%
        \def\PYZsq{\textquotesingle}% Upright quotes in Pygmentized code
    }
    \usepackage{upquote} % Upright quotes for verbatim code
    \usepackage{eurosym} % defines \euro
    \usepackage[mathletters]{ucs} % Extended unicode (utf-8) support
    \usepackage[utf8x]{inputenc} % Allow utf-8 characters in the tex document
    \usepackage{fancyvrb} % verbatim replacement that allows latex
    \usepackage{grffile} % extends the file name processing of package graphics 
                         % to support a larger range 
    % The hyperref package gives us a pdf with properly built
    % internal navigation ('pdf bookmarks' for the table of contents,
    % internal cross-reference links, web links for URLs, etc.)
    \usepackage{hyperref}
    \usepackage{longtable} % longtable support required by pandoc >1.10
    \usepackage{booktabs}  % table support for pandoc > 1.12.2
    \usepackage[inline]{enumitem} % IRkernel/repr support (it uses the enumerate* environment)
    \usepackage[normalem]{ulem} % ulem is needed to support strikethroughs (\sout)
                                % normalem makes italics be italics, not underlines
    

    
    
    % Colors for the hyperref package
    \definecolor{urlcolor}{rgb}{0,.145,.698}
    \definecolor{linkcolor}{rgb}{.71,0.21,0.01}
    \definecolor{citecolor}{rgb}{.12,.54,.11}

    % ANSI colors
    \definecolor{ansi-black}{HTML}{3E424D}
    \definecolor{ansi-black-intense}{HTML}{282C36}
    \definecolor{ansi-red}{HTML}{E75C58}
    \definecolor{ansi-red-intense}{HTML}{B22B31}
    \definecolor{ansi-green}{HTML}{00A250}
    \definecolor{ansi-green-intense}{HTML}{007427}
    \definecolor{ansi-yellow}{HTML}{DDB62B}
    \definecolor{ansi-yellow-intense}{HTML}{B27D12}
    \definecolor{ansi-blue}{HTML}{208FFB}
    \definecolor{ansi-blue-intense}{HTML}{0065CA}
    \definecolor{ansi-magenta}{HTML}{D160C4}
    \definecolor{ansi-magenta-intense}{HTML}{A03196}
    \definecolor{ansi-cyan}{HTML}{60C6C8}
    \definecolor{ansi-cyan-intense}{HTML}{258F8F}
    \definecolor{ansi-white}{HTML}{C5C1B4}
    \definecolor{ansi-white-intense}{HTML}{A1A6B2}

    % commands and environments needed by pandoc snippets
    % extracted from the output of `pandoc -s`
    \providecommand{\tightlist}{%
      \setlength{\itemsep}{0pt}\setlength{\parskip}{0pt}}
    \DefineVerbatimEnvironment{Highlighting}{Verbatim}{commandchars=\\\{\}}
    % Add ',fontsize=\small' for more characters per line
    \newenvironment{Shaded}{}{}
    \newcommand{\KeywordTok}[1]{\textcolor[rgb]{0.00,0.44,0.13}{\textbf{{#1}}}}
    \newcommand{\DataTypeTok}[1]{\textcolor[rgb]{0.56,0.13,0.00}{{#1}}}
    \newcommand{\DecValTok}[1]{\textcolor[rgb]{0.25,0.63,0.44}{{#1}}}
    \newcommand{\BaseNTok}[1]{\textcolor[rgb]{0.25,0.63,0.44}{{#1}}}
    \newcommand{\FloatTok}[1]{\textcolor[rgb]{0.25,0.63,0.44}{{#1}}}
    \newcommand{\CharTok}[1]{\textcolor[rgb]{0.25,0.44,0.63}{{#1}}}
    \newcommand{\StringTok}[1]{\textcolor[rgb]{0.25,0.44,0.63}{{#1}}}
    \newcommand{\CommentTok}[1]{\textcolor[rgb]{0.38,0.63,0.69}{\textit{{#1}}}}
    \newcommand{\OtherTok}[1]{\textcolor[rgb]{0.00,0.44,0.13}{{#1}}}
    \newcommand{\AlertTok}[1]{\textcolor[rgb]{1.00,0.00,0.00}{\textbf{{#1}}}}
    \newcommand{\FunctionTok}[1]{\textcolor[rgb]{0.02,0.16,0.49}{{#1}}}
    \newcommand{\RegionMarkerTok}[1]{{#1}}
    \newcommand{\ErrorTok}[1]{\textcolor[rgb]{1.00,0.00,0.00}{\textbf{{#1}}}}
    \newcommand{\NormalTok}[1]{{#1}}
    
    % Additional commands for more recent versions of Pandoc
    \newcommand{\ConstantTok}[1]{\textcolor[rgb]{0.53,0.00,0.00}{{#1}}}
    \newcommand{\SpecialCharTok}[1]{\textcolor[rgb]{0.25,0.44,0.63}{{#1}}}
    \newcommand{\VerbatimStringTok}[1]{\textcolor[rgb]{0.25,0.44,0.63}{{#1}}}
    \newcommand{\SpecialStringTok}[1]{\textcolor[rgb]{0.73,0.40,0.53}{{#1}}}
    \newcommand{\ImportTok}[1]{{#1}}
    \newcommand{\DocumentationTok}[1]{\textcolor[rgb]{0.73,0.13,0.13}{\textit{{#1}}}}
    \newcommand{\AnnotationTok}[1]{\textcolor[rgb]{0.38,0.63,0.69}{\textbf{\textit{{#1}}}}}
    \newcommand{\CommentVarTok}[1]{\textcolor[rgb]{0.38,0.63,0.69}{\textbf{\textit{{#1}}}}}
    \newcommand{\VariableTok}[1]{\textcolor[rgb]{0.10,0.09,0.49}{{#1}}}
    \newcommand{\ControlFlowTok}[1]{\textcolor[rgb]{0.00,0.44,0.13}{\textbf{{#1}}}}
    \newcommand{\OperatorTok}[1]{\textcolor[rgb]{0.40,0.40,0.40}{{#1}}}
    \newcommand{\BuiltInTok}[1]{{#1}}
    \newcommand{\ExtensionTok}[1]{{#1}}
    \newcommand{\PreprocessorTok}[1]{\textcolor[rgb]{0.74,0.48,0.00}{{#1}}}
    \newcommand{\AttributeTok}[1]{\textcolor[rgb]{0.49,0.56,0.16}{{#1}}}
    \newcommand{\InformationTok}[1]{\textcolor[rgb]{0.38,0.63,0.69}{\textbf{\textit{{#1}}}}}
    \newcommand{\WarningTok}[1]{\textcolor[rgb]{0.38,0.63,0.69}{\textbf{\textit{{#1}}}}}
    
    
    % Define a nice break command that doesn't care if a line doesn't already
    % exist.
    \def\br{\hspace*{\fill} \\* }
    % Math Jax compatability definitions
    \def\gt{>}
    \def\lt{<}
    % Document parameters
    \title{PyCitySchools-report\_pdf}
    
    
    

    % Pygments definitions
    
\makeatletter
\def\PY@reset{\let\PY@it=\relax \let\PY@bf=\relax%
    \let\PY@ul=\relax \let\PY@tc=\relax%
    \let\PY@bc=\relax \let\PY@ff=\relax}
\def\PY@tok#1{\csname PY@tok@#1\endcsname}
\def\PY@toks#1+{\ifx\relax#1\empty\else%
    \PY@tok{#1}\expandafter\PY@toks\fi}
\def\PY@do#1{\PY@bc{\PY@tc{\PY@ul{%
    \PY@it{\PY@bf{\PY@ff{#1}}}}}}}
\def\PY#1#2{\PY@reset\PY@toks#1+\relax+\PY@do{#2}}

\expandafter\def\csname PY@tok@w\endcsname{\def\PY@tc##1{\textcolor[rgb]{0.73,0.73,0.73}{##1}}}
\expandafter\def\csname PY@tok@c\endcsname{\let\PY@it=\textit\def\PY@tc##1{\textcolor[rgb]{0.25,0.50,0.50}{##1}}}
\expandafter\def\csname PY@tok@cp\endcsname{\def\PY@tc##1{\textcolor[rgb]{0.74,0.48,0.00}{##1}}}
\expandafter\def\csname PY@tok@k\endcsname{\let\PY@bf=\textbf\def\PY@tc##1{\textcolor[rgb]{0.00,0.50,0.00}{##1}}}
\expandafter\def\csname PY@tok@kp\endcsname{\def\PY@tc##1{\textcolor[rgb]{0.00,0.50,0.00}{##1}}}
\expandafter\def\csname PY@tok@kt\endcsname{\def\PY@tc##1{\textcolor[rgb]{0.69,0.00,0.25}{##1}}}
\expandafter\def\csname PY@tok@o\endcsname{\def\PY@tc##1{\textcolor[rgb]{0.40,0.40,0.40}{##1}}}
\expandafter\def\csname PY@tok@ow\endcsname{\let\PY@bf=\textbf\def\PY@tc##1{\textcolor[rgb]{0.67,0.13,1.00}{##1}}}
\expandafter\def\csname PY@tok@nb\endcsname{\def\PY@tc##1{\textcolor[rgb]{0.00,0.50,0.00}{##1}}}
\expandafter\def\csname PY@tok@nf\endcsname{\def\PY@tc##1{\textcolor[rgb]{0.00,0.00,1.00}{##1}}}
\expandafter\def\csname PY@tok@nc\endcsname{\let\PY@bf=\textbf\def\PY@tc##1{\textcolor[rgb]{0.00,0.00,1.00}{##1}}}
\expandafter\def\csname PY@tok@nn\endcsname{\let\PY@bf=\textbf\def\PY@tc##1{\textcolor[rgb]{0.00,0.00,1.00}{##1}}}
\expandafter\def\csname PY@tok@ne\endcsname{\let\PY@bf=\textbf\def\PY@tc##1{\textcolor[rgb]{0.82,0.25,0.23}{##1}}}
\expandafter\def\csname PY@tok@nv\endcsname{\def\PY@tc##1{\textcolor[rgb]{0.10,0.09,0.49}{##1}}}
\expandafter\def\csname PY@tok@no\endcsname{\def\PY@tc##1{\textcolor[rgb]{0.53,0.00,0.00}{##1}}}
\expandafter\def\csname PY@tok@nl\endcsname{\def\PY@tc##1{\textcolor[rgb]{0.63,0.63,0.00}{##1}}}
\expandafter\def\csname PY@tok@ni\endcsname{\let\PY@bf=\textbf\def\PY@tc##1{\textcolor[rgb]{0.60,0.60,0.60}{##1}}}
\expandafter\def\csname PY@tok@na\endcsname{\def\PY@tc##1{\textcolor[rgb]{0.49,0.56,0.16}{##1}}}
\expandafter\def\csname PY@tok@nt\endcsname{\let\PY@bf=\textbf\def\PY@tc##1{\textcolor[rgb]{0.00,0.50,0.00}{##1}}}
\expandafter\def\csname PY@tok@nd\endcsname{\def\PY@tc##1{\textcolor[rgb]{0.67,0.13,1.00}{##1}}}
\expandafter\def\csname PY@tok@s\endcsname{\def\PY@tc##1{\textcolor[rgb]{0.73,0.13,0.13}{##1}}}
\expandafter\def\csname PY@tok@sd\endcsname{\let\PY@it=\textit\def\PY@tc##1{\textcolor[rgb]{0.73,0.13,0.13}{##1}}}
\expandafter\def\csname PY@tok@si\endcsname{\let\PY@bf=\textbf\def\PY@tc##1{\textcolor[rgb]{0.73,0.40,0.53}{##1}}}
\expandafter\def\csname PY@tok@se\endcsname{\let\PY@bf=\textbf\def\PY@tc##1{\textcolor[rgb]{0.73,0.40,0.13}{##1}}}
\expandafter\def\csname PY@tok@sr\endcsname{\def\PY@tc##1{\textcolor[rgb]{0.73,0.40,0.53}{##1}}}
\expandafter\def\csname PY@tok@ss\endcsname{\def\PY@tc##1{\textcolor[rgb]{0.10,0.09,0.49}{##1}}}
\expandafter\def\csname PY@tok@sx\endcsname{\def\PY@tc##1{\textcolor[rgb]{0.00,0.50,0.00}{##1}}}
\expandafter\def\csname PY@tok@m\endcsname{\def\PY@tc##1{\textcolor[rgb]{0.40,0.40,0.40}{##1}}}
\expandafter\def\csname PY@tok@gh\endcsname{\let\PY@bf=\textbf\def\PY@tc##1{\textcolor[rgb]{0.00,0.00,0.50}{##1}}}
\expandafter\def\csname PY@tok@gu\endcsname{\let\PY@bf=\textbf\def\PY@tc##1{\textcolor[rgb]{0.50,0.00,0.50}{##1}}}
\expandafter\def\csname PY@tok@gd\endcsname{\def\PY@tc##1{\textcolor[rgb]{0.63,0.00,0.00}{##1}}}
\expandafter\def\csname PY@tok@gi\endcsname{\def\PY@tc##1{\textcolor[rgb]{0.00,0.63,0.00}{##1}}}
\expandafter\def\csname PY@tok@gr\endcsname{\def\PY@tc##1{\textcolor[rgb]{1.00,0.00,0.00}{##1}}}
\expandafter\def\csname PY@tok@ge\endcsname{\let\PY@it=\textit}
\expandafter\def\csname PY@tok@gs\endcsname{\let\PY@bf=\textbf}
\expandafter\def\csname PY@tok@gp\endcsname{\let\PY@bf=\textbf\def\PY@tc##1{\textcolor[rgb]{0.00,0.00,0.50}{##1}}}
\expandafter\def\csname PY@tok@go\endcsname{\def\PY@tc##1{\textcolor[rgb]{0.53,0.53,0.53}{##1}}}
\expandafter\def\csname PY@tok@gt\endcsname{\def\PY@tc##1{\textcolor[rgb]{0.00,0.27,0.87}{##1}}}
\expandafter\def\csname PY@tok@err\endcsname{\def\PY@bc##1{\setlength{\fboxsep}{0pt}\fcolorbox[rgb]{1.00,0.00,0.00}{1,1,1}{\strut ##1}}}
\expandafter\def\csname PY@tok@kc\endcsname{\let\PY@bf=\textbf\def\PY@tc##1{\textcolor[rgb]{0.00,0.50,0.00}{##1}}}
\expandafter\def\csname PY@tok@kd\endcsname{\let\PY@bf=\textbf\def\PY@tc##1{\textcolor[rgb]{0.00,0.50,0.00}{##1}}}
\expandafter\def\csname PY@tok@kn\endcsname{\let\PY@bf=\textbf\def\PY@tc##1{\textcolor[rgb]{0.00,0.50,0.00}{##1}}}
\expandafter\def\csname PY@tok@kr\endcsname{\let\PY@bf=\textbf\def\PY@tc##1{\textcolor[rgb]{0.00,0.50,0.00}{##1}}}
\expandafter\def\csname PY@tok@bp\endcsname{\def\PY@tc##1{\textcolor[rgb]{0.00,0.50,0.00}{##1}}}
\expandafter\def\csname PY@tok@fm\endcsname{\def\PY@tc##1{\textcolor[rgb]{0.00,0.00,1.00}{##1}}}
\expandafter\def\csname PY@tok@vc\endcsname{\def\PY@tc##1{\textcolor[rgb]{0.10,0.09,0.49}{##1}}}
\expandafter\def\csname PY@tok@vg\endcsname{\def\PY@tc##1{\textcolor[rgb]{0.10,0.09,0.49}{##1}}}
\expandafter\def\csname PY@tok@vi\endcsname{\def\PY@tc##1{\textcolor[rgb]{0.10,0.09,0.49}{##1}}}
\expandafter\def\csname PY@tok@vm\endcsname{\def\PY@tc##1{\textcolor[rgb]{0.10,0.09,0.49}{##1}}}
\expandafter\def\csname PY@tok@sa\endcsname{\def\PY@tc##1{\textcolor[rgb]{0.73,0.13,0.13}{##1}}}
\expandafter\def\csname PY@tok@sb\endcsname{\def\PY@tc##1{\textcolor[rgb]{0.73,0.13,0.13}{##1}}}
\expandafter\def\csname PY@tok@sc\endcsname{\def\PY@tc##1{\textcolor[rgb]{0.73,0.13,0.13}{##1}}}
\expandafter\def\csname PY@tok@dl\endcsname{\def\PY@tc##1{\textcolor[rgb]{0.73,0.13,0.13}{##1}}}
\expandafter\def\csname PY@tok@s2\endcsname{\def\PY@tc##1{\textcolor[rgb]{0.73,0.13,0.13}{##1}}}
\expandafter\def\csname PY@tok@sh\endcsname{\def\PY@tc##1{\textcolor[rgb]{0.73,0.13,0.13}{##1}}}
\expandafter\def\csname PY@tok@s1\endcsname{\def\PY@tc##1{\textcolor[rgb]{0.73,0.13,0.13}{##1}}}
\expandafter\def\csname PY@tok@mb\endcsname{\def\PY@tc##1{\textcolor[rgb]{0.40,0.40,0.40}{##1}}}
\expandafter\def\csname PY@tok@mf\endcsname{\def\PY@tc##1{\textcolor[rgb]{0.40,0.40,0.40}{##1}}}
\expandafter\def\csname PY@tok@mh\endcsname{\def\PY@tc##1{\textcolor[rgb]{0.40,0.40,0.40}{##1}}}
\expandafter\def\csname PY@tok@mi\endcsname{\def\PY@tc##1{\textcolor[rgb]{0.40,0.40,0.40}{##1}}}
\expandafter\def\csname PY@tok@il\endcsname{\def\PY@tc##1{\textcolor[rgb]{0.40,0.40,0.40}{##1}}}
\expandafter\def\csname PY@tok@mo\endcsname{\def\PY@tc##1{\textcolor[rgb]{0.40,0.40,0.40}{##1}}}
\expandafter\def\csname PY@tok@ch\endcsname{\let\PY@it=\textit\def\PY@tc##1{\textcolor[rgb]{0.25,0.50,0.50}{##1}}}
\expandafter\def\csname PY@tok@cm\endcsname{\let\PY@it=\textit\def\PY@tc##1{\textcolor[rgb]{0.25,0.50,0.50}{##1}}}
\expandafter\def\csname PY@tok@cpf\endcsname{\let\PY@it=\textit\def\PY@tc##1{\textcolor[rgb]{0.25,0.50,0.50}{##1}}}
\expandafter\def\csname PY@tok@c1\endcsname{\let\PY@it=\textit\def\PY@tc##1{\textcolor[rgb]{0.25,0.50,0.50}{##1}}}
\expandafter\def\csname PY@tok@cs\endcsname{\let\PY@it=\textit\def\PY@tc##1{\textcolor[rgb]{0.25,0.50,0.50}{##1}}}

\def\PYZbs{\char`\\}
\def\PYZus{\char`\_}
\def\PYZob{\char`\{}
\def\PYZcb{\char`\}}
\def\PYZca{\char`\^}
\def\PYZam{\char`\&}
\def\PYZlt{\char`\<}
\def\PYZgt{\char`\>}
\def\PYZsh{\char`\#}
\def\PYZpc{\char`\%}
\def\PYZdl{\char`\$}
\def\PYZhy{\char`\-}
\def\PYZsq{\char`\'}
\def\PYZdq{\char`\"}
\def\PYZti{\char`\~}
% for compatibility with earlier versions
\def\PYZat{@}
\def\PYZlb{[}
\def\PYZrb{]}
\makeatother


    % Exact colors from NB
    \definecolor{incolor}{rgb}{0.0, 0.0, 0.5}
    \definecolor{outcolor}{rgb}{0.545, 0.0, 0.0}



    
    % Prevent overflowing lines due to hard-to-break entities
    \sloppy 
    % Setup hyperref package
    \hypersetup{
      breaklinks=true,  % so long urls are correctly broken across lines
      colorlinks=true,
      urlcolor=urlcolor,
      linkcolor=linkcolor,
      citecolor=citecolor,
      }
    % Slightly bigger margins than the latex defaults
    
    \geometry{verbose,tmargin=1in,bmargin=1in,lmargin=1in,rmargin=1in}
    
    

    \begin{document}
    
    
    \maketitle
    
    

    
    \subsection{PYCITYSCHOOLS}\label{pycityschools}

    \begin{Verbatim}[commandchars=\\\{\}]
{\color{incolor}In [{\color{incolor}1}]:} \PY{c+c1}{\PYZsh{}import the neccessary packages}
        \PY{k+kn}{import} \PY{n+nn}{pandas} \PY{k}{as} \PY{n+nn}{pd}
        \PY{k+kn}{import} \PY{n+nn}{matplotlib}\PY{n+nn}{.}\PY{n+nn}{pyplot} \PY{k}{as} \PY{n+nn}{plt}
\end{Verbatim}


    \begin{Verbatim}[commandchars=\\\{\}]
{\color{incolor}In [{\color{incolor}4}]:} \PY{n}{school\PYZus{}df}\PY{o}{.}\PY{n}{head}\PY{p}{(}\PY{p}{)}
\end{Verbatim}


\begin{Verbatim}[commandchars=\\\{\}]
{\color{outcolor}Out[{\color{outcolor}4}]:}    School ID                   name      type  size   budget
        0          0      Huang High School  District  2917  1910635
        1          1   Figueroa High School  District  2949  1884411
        2          2    Shelton High School   Charter  1761  1056600
        3          3  Hernandez High School  District  4635  3022020
        4          4    Griffin High School   Charter  1468   917500
\end{Verbatim}
            
    \begin{Verbatim}[commandchars=\\\{\}]
{\color{incolor}In [{\color{incolor}5}]:} \PY{n}{student\PYZus{}df}\PY{o}{.}\PY{n}{head}\PY{p}{(}\PY{p}{)}
\end{Verbatim}


\begin{Verbatim}[commandchars=\\\{\}]
{\color{outcolor}Out[{\color{outcolor}5}]:}    Student ID               name gender grade             school  \textbackslash{}
        0           0       Paul Bradley      M   9th  Huang High School   
        1           1       Victor Smith      M  12th  Huang High School   
        2           2    Kevin Rodriguez      M  12th  Huang High School   
        3           3  Dr. Richard Scott      M  12th  Huang High School   
        4           4         Bonnie Ray      F   9th  Huang High School   
        
           reading\_score  math\_score  
        0             66          79  
        1             94          61  
        2             90          60  
        3             67          58  
        4             97          84  
\end{Verbatim}
            
    \begin{Verbatim}[commandchars=\\\{\}]
{\color{incolor}In [{\color{incolor}6}]:} \PY{c+c1}{\PYZsh{} rename the column }
        \PY{n}{school\PYZus{}df}\PY{o}{=}\PY{n}{school\PYZus{}df}\PY{o}{.}\PY{n}{rename}\PY{p}{(}\PY{n}{columns}\PY{o}{=}\PY{p}{\PYZob{}}\PY{l+s+s2}{\PYZdq{}}\PY{l+s+s2}{name}\PY{l+s+s2}{\PYZdq{}}\PY{p}{:}\PY{l+s+s2}{\PYZdq{}}\PY{l+s+s2}{school}\PY{l+s+s2}{\PYZdq{}}\PY{p}{\PYZcb{}}\PY{p}{)}
        \PY{n}{school\PYZus{}df}\PY{o}{.}\PY{n}{head}\PY{p}{(}\PY{p}{)}
\end{Verbatim}


\begin{Verbatim}[commandchars=\\\{\}]
{\color{outcolor}Out[{\color{outcolor}6}]:}    School ID                 school      type  size   budget
        0          0      Huang High School  District  2917  1910635
        1          1   Figueroa High School  District  2949  1884411
        2          2    Shelton High School   Charter  1761  1056600
        3          3  Hernandez High School  District  4635  3022020
        4          4    Griffin High School   Charter  1468   917500
\end{Verbatim}
            
    Now the column names are same, perform merge, on school

    \begin{Verbatim}[commandchars=\\\{\}]
{\color{incolor}In [{\color{incolor}7}]:} \PY{n}{student\PYZus{}school\PYZus{}df}\PY{o}{=}\PY{n}{pd}\PY{o}{.}\PY{n}{merge}\PY{p}{(}\PY{n}{student\PYZus{}df}\PY{p}{,}\PY{n}{school\PYZus{}df}\PY{p}{,}\PY{n}{on}\PY{o}{=}\PY{l+s+s2}{\PYZdq{}}\PY{l+s+s2}{school}\PY{l+s+s2}{\PYZdq{}}\PY{p}{)}
\end{Verbatim}


    \begin{Verbatim}[commandchars=\\\{\}]
{\color{incolor}In [{\color{incolor}8}]:} \PY{n}{student\PYZus{}school\PYZus{}df}\PY{o}{.}\PY{n}{head}\PY{p}{(}\PY{p}{)}
\end{Verbatim}


\begin{Verbatim}[commandchars=\\\{\}]
{\color{outcolor}Out[{\color{outcolor}8}]:}    Student ID               name gender grade             school  \textbackslash{}
        0           0       Paul Bradley      M   9th  Huang High School   
        1           1       Victor Smith      M  12th  Huang High School   
        2           2    Kevin Rodriguez      M  12th  Huang High School   
        3           3  Dr. Richard Scott      M  12th  Huang High School   
        4           4         Bonnie Ray      F   9th  Huang High School   
        
           reading\_score  math\_score  School ID      type  size   budget  
        0             66          79          0  District  2917  1910635  
        1             94          61          0  District  2917  1910635  
        2             90          60          0  District  2917  1910635  
        3             67          58          0  District  2917  1910635  
        4             97          84          0  District  2917  1910635  
\end{Verbatim}
            
    \subsection{District}\label{district}

    \begin{Verbatim}[commandchars=\\\{\}]
{\color{incolor}In [{\color{incolor}11}]:} \PY{n}{summary\PYZus{}of\PYZus{}District\PYZus{}Schools}\PY{o}{=}\PY{n}{district\PYZus{}summary}\PY{p}{(}\PY{p}{)}
         \PY{n}{summary\PYZus{}of\PYZus{}District\PYZus{}Schools}
\end{Verbatim}


\begin{Verbatim}[commandchars=\\\{\}]
{\color{outcolor}Out[{\color{outcolor}11}]:}    School Count  student Count    Budget  Avg Maths Score  Avg Reading Score  \textbackslash{}
         0             7          26976  17347923            76.99              80.96   
         
            Maths Pass \%  Reading Pass \%  Overall Pass \%  
         0         64.31           78.37           71.34  
\end{Verbatim}
            
    \paragraph{From above summary , it can be seen that, 64\% of students
passed Maths and 78.37\% passed reading and Overall \% is
71.34}\label{from-above-summary-it-can-be-seen-that-64-of-students-passed-maths-and-78.37-passed-reading-and-overall-is-71.34}

    \subsection{Pass \% by School}\label{pass-by-school}

    \begin{Verbatim}[commandchars=\\\{\}]
{\color{incolor}In [{\color{incolor}13}]:} \PY{c+c1}{\PYZsh{}use the function to get summary}
         \PY{n}{school\PYZus{}tot}\PY{o}{=}\PY{n}{passSchool}\PY{p}{(}\PY{p}{)}
         \PY{n}{school\PYZus{}tot}\PY{o}{.}\PY{n}{head}\PY{p}{(}\PY{p}{)}
\end{Verbatim}


\begin{Verbatim}[commandchars=\\\{\}]
{\color{outcolor}Out[{\color{outcolor}13}]:}    School ID            School Name      Type   Budget  Budget Per Student  \textbackslash{}
         0          0      Huang High School  District  1910635               655.0   
         1          1   Figueroa High School  District  1884411               639.0   
         2          2    Shelton High School   Charter  1056600               600.0   
         3          3  Hernandez High School  District  3022020               652.0   
         4          4    Griffin High School   Charter   917500               625.0   
         
            Student Count  Avg Math Score  Avg Read Score  Math \%  Read \%  Overall \%  
         0           2917           76.63           81.18   63.32   78.81      71.06  
         1           2949           76.71           81.16   63.75   78.43      71.09  
         2           1761           83.36           83.73   89.89   92.62      91.25  
         3           4635           77.29           80.93   64.75   78.19      71.47  
         4           1468           83.35           83.82   89.71   93.39      91.55  
\end{Verbatim}
            
    \paragraph{Top 5 Schools}\label{top-5-schools}

    \begin{Verbatim}[commandchars=\\\{\}]
{\color{incolor}In [{\color{incolor}14}]:} \PY{c+c1}{\PYZsh{}sort the schools and display 5 top schools based on Overall \PYZpc{}}
         \PY{n}{sort\PYZus{}school\PYZus{}top}\PY{o}{=}\PY{n}{school\PYZus{}tot}\PY{o}{.}\PY{n}{sort\PYZus{}values}\PY{p}{(}\PY{p}{[}\PY{l+s+s2}{\PYZdq{}}\PY{l+s+s2}{Overall }\PY{l+s+s2}{\PYZpc{}}\PY{l+s+s2}{\PYZdq{}}\PY{p}{]}\PY{p}{,}\PY{n}{ascending}\PY{o}{=}\PY{k+kc}{False}\PY{p}{)}
         \PY{n}{sort\PYZus{}school\PYZus{}top}\PY{o}{.}\PY{n}{head}\PY{p}{(}\PY{p}{)}
\end{Verbatim}


\begin{Verbatim}[commandchars=\\\{\}]
{\color{outcolor}Out[{\color{outcolor}14}]:}     School ID          School Name     Type   Budget  Budget Per Student  \textbackslash{}
         5           5   Wilson High School  Charter  1319574               578.0   
         9           9     Pena High School  Charter   585858               609.0   
         10         10   Wright High School  Charter  1049400               583.0   
         6           6  Cabrera High School  Charter  1081356               582.0   
         8           8   Holden High School  Charter   248087               581.0   
         
             Student Count  Avg Math Score  Avg Read Score  Math \%  Read \%  Overall \%  
         5            2283           83.27           83.99   90.93   93.25      92.09  
         9             962           83.84           84.04   91.68   92.20      91.94  
         10           1800           83.68           83.95   90.28   93.44      91.86  
         6            1858           83.06           83.98   89.56   93.86      91.71  
         8             427           83.80           83.81   90.63   92.74      91.69  
\end{Verbatim}
            
    \paragraph{\texorpdfstring{The top school is Wilson High School with a
Overall pass \% of 92.09. It can be seen that , all schools in the top 5
are type: Charter. Also pass \% for Maths is lower than reading \% for
schools.}{The top school is  Wilson High School  with a Overall pass \% of 92.09. It can be seen that , all schools in the top 5 are type: Charter. Also pass \% for Maths is lower than reading \% for schools.}}\label{the-top-school-is-wilson-high-school-with-a-overall-pass-of-92.09.-it-can-be-seen-that-all-schools-in-the-top-5-are-type-charter.-also-pass-for-maths-is-lower-than-reading-for-schools.}

    \paragraph{Bottom 5 Schools}\label{bottom-5-schools}

    \begin{Verbatim}[commandchars=\\\{\}]
{\color{incolor}In [{\color{incolor}15}]:} \PY{c+c1}{\PYZsh{}sort the schools and display 5 bottom schools based on Overall \PYZpc{}}
         \PY{n}{sort\PYZus{}school\PYZus{}bottom}\PY{o}{=}\PY{n}{school\PYZus{}tot}\PY{o}{.}\PY{n}{sort\PYZus{}values}\PY{p}{(}\PY{p}{[}\PY{l+s+s2}{\PYZdq{}}\PY{l+s+s2}{Overall }\PY{l+s+s2}{\PYZpc{}}\PY{l+s+s2}{\PYZdq{}}\PY{p}{]}\PY{p}{,}\PY{n}{ascending}\PY{o}{=}\PY{k+kc}{True}\PY{p}{)}
         \PY{n}{sort\PYZus{}school\PYZus{}bottom}\PY{o}{.}\PY{n}{head}\PY{p}{(}\PY{p}{)}
\end{Verbatim}


\begin{Verbatim}[commandchars=\\\{\}]
{\color{outcolor}Out[{\color{outcolor}15}]:}     School ID            School Name      Type   Budget  Budget Per Student  \textbackslash{}
         11         11  Rodriguez High School  District  2547363               637.0   
         0           0      Huang High School  District  1910635               655.0   
         12         12    Johnson High School  District  3094650               650.0   
         1           1   Figueroa High School  District  1884411               639.0   
         3           3  Hernandez High School  District  3022020               652.0   
         
             Student Count  Avg Math Score  Avg Read Score  Math \%  Read \%  Overall \%  
         11           3999           76.84           80.74   64.07   77.74      70.91  
         0            2917           76.63           81.18   63.32   78.81      71.06  
         12           4761           77.07           80.97   63.85   78.28      71.06  
         1            2949           76.71           81.16   63.75   78.43      71.09  
         3            4635           77.29           80.93   64.75   78.19      71.47  
\end{Verbatim}
            
    \paragraph{\texorpdfstring{The Rodriguez High School has the lowest
Overall pass \% 70.91\%. It can be seen that , all schools in the bottom
5 are type: District . Also pass \% for Maths is lower than reading \%
for
schools.}{The  Rodriguez High School  has the lowest Overall pass \% 70.91\%. It can be seen that , all schools in the bottom 5 are type: District . Also pass \% for Maths is lower than reading \% for schools.}}\label{the-rodriguez-high-school-has-the-lowest-overall-pass-70.91.-it-can-be-seen-that-all-schools-in-the-bottom-5-are-type-district-.-also-pass-for-maths-is-lower-than-reading-for-schools.}

    \subsection{Score by Grade}\label{score-by-grade}

    \begin{Verbatim}[commandchars=\\\{\}]
{\color{incolor}In [{\color{incolor}17}]:} \PY{n}{grade\PYZus{}math\PYZus{}df}\PY{o}{=}\PY{n}{grade\PYZus{}sort}\PY{p}{(}\PY{l+s+s2}{\PYZdq{}}\PY{l+s+s2}{math\PYZus{}score}\PY{l+s+s2}{\PYZdq{}}\PY{p}{)}
         \PY{n}{grade\PYZus{}read\PYZus{}df}\PY{o}{=}\PY{n}{grade\PYZus{}sort}\PY{p}{(}\PY{l+s+s2}{\PYZdq{}}\PY{l+s+s2}{reading\PYZus{}score}\PY{l+s+s2}{\PYZdq{}}\PY{p}{)}
\end{Verbatim}


    \subparagraph{AVERAGE MATH SCORE BY
GRADE}\label{average-math-score-by-grade}

    \begin{Verbatim}[commandchars=\\\{\}]
{\color{incolor}In [{\color{incolor}18}]:} \PY{n}{grade\PYZus{}math\PYZus{}df}\PY{o}{.}\PY{n}{head}\PY{p}{(}\PY{p}{)}
\end{Verbatim}


\begin{Verbatim}[commandchars=\\\{\}]
{\color{outcolor}Out[{\color{outcolor}18}]:}              School Name    9th   10th   11th   12th
         0      Huang High School  77.03  75.91  76.45  77.23
         1   Figueroa High School  76.40  76.54  76.88  77.15
         2    Shelton High School  83.42  82.92  83.38  83.78
         3  Hernandez High School  77.44  77.34  77.14  77.19
         4    Griffin High School  82.04  84.23  83.84  83.36
\end{Verbatim}
            
    \paragraph{The 9th grade students have scored maths average score higher
than other grades and the 10th grade students have lowest maths average
scores}\label{the-9th-grade-students-have-scored-maths-average-score-higher-than-other-grades-and-the-10th-grade-students-have-lowest-maths-average-scores}

    \subparagraph{AVERAGE READING SCORE BY
GRADE}\label{average-reading-score-by-grade}

    \begin{Verbatim}[commandchars=\\\{\}]
{\color{incolor}In [{\color{incolor}19}]:} \PY{n}{grade\PYZus{}read\PYZus{}df}\PY{o}{.}\PY{n}{head}\PY{p}{(}\PY{p}{)}
\end{Verbatim}


\begin{Verbatim}[commandchars=\\\{\}]
{\color{outcolor}Out[{\color{outcolor}19}]:}              School Name    9th   10th   11th   12th
         0      Huang High School  81.29  81.51  81.42  80.31
         1   Figueroa High School  81.20  81.41  80.64  81.38
         2    Shelton High School  84.12  83.44  84.37  82.78
         3  Hernandez High School  80.87  80.66  81.40  80.86
         4    Griffin High School  83.37  83.71  84.29  84.01
\end{Verbatim}
            
    \paragraph{The 10th grade students have scored reading average score
higher than other grades for most schools and the 12th grade students
have lowest maths average scores for most
schools.}\label{the-10th-grade-students-have-scored-reading-average-score-higher-than-other-grades-for-most-schools-and-the-12th-grade-students-have-lowest-maths-average-scores-for-most-schools.}

    \subsubsection{SCORE BY SCHOOL SPENDING}\label{score-by-school-spending}

    \begin{Verbatim}[commandchars=\\\{\}]
{\color{incolor}In [{\color{incolor}21}]:} \PY{c+c1}{\PYZsh{} call the function to retrieve df }
         \PY{n}{spend\PYZus{}df}\PY{o}{=}\PY{n}{spending\PYZus{}student}\PY{p}{(}\PY{p}{)}
         \PY{c+c1}{\PYZsh{}create bins}
         \PY{n}{bins}\PY{o}{=}\PY{p}{[}\PY{l+m+mi}{0}\PY{p}{,}\PY{l+m+mi}{585}\PY{p}{,} \PY{l+m+mi}{615}\PY{p}{,} \PY{l+m+mi}{645}\PY{p}{,} \PY{l+m+mi}{675}\PY{p}{]}
         \PY{c+c1}{\PYZsh{} Create the names for the four bins}
         \PY{n}{spend\PYZus{}labels} \PY{o}{=} \PY{p}{[}\PY{l+s+s1}{\PYZsq{}}\PY{l+s+s1}{\PYZlt{}585}\PY{l+s+s1}{\PYZsq{}}\PY{p}{,} \PY{l+s+s1}{\PYZsq{}}\PY{l+s+s1}{585\PYZhy{}615}\PY{l+s+s1}{\PYZsq{}}\PY{p}{,} \PY{l+s+s1}{\PYZsq{}}\PY{l+s+s1}{616\PYZhy{}645}\PY{l+s+s1}{\PYZsq{}}\PY{p}{,} \PY{l+s+s1}{\PYZsq{}}\PY{l+s+s1}{646\PYZhy{}675}\PY{l+s+s1}{\PYZsq{}}\PY{p}{]}
         \PY{c+c1}{\PYZsh{} cut bins according to values}
         \PY{n}{spend\PYZus{}df}\PY{p}{[}\PY{l+s+s2}{\PYZdq{}}\PY{l+s+s2}{Spend Group}\PY{l+s+s2}{\PYZdq{}}\PY{p}{]} \PY{o}{=} \PY{n}{pd}\PY{o}{.}\PY{n}{cut}\PY{p}{(}\PY{n}{spend\PYZus{}df}\PY{p}{[}\PY{l+s+s2}{\PYZdq{}}\PY{l+s+s2}{Budget per Student}\PY{l+s+s2}{\PYZdq{}}\PY{p}{]}\PY{p}{,}\PY{n}{bins}\PY{p}{,}\PY{n}{labels}\PY{o}{=}\PY{n}{spend\PYZus{}labels} \PY{p}{)}
         \PY{n}{spend\PYZus{}df}\PY{o}{.}\PY{n}{head}\PY{p}{(}\PY{p}{)}
\end{Verbatim}


\begin{Verbatim}[commandchars=\\\{\}]
{\color{outcolor}Out[{\color{outcolor}21}]:}    School ID      Type  Student Count            School Name  School Budget  \textbackslash{}
         0          0  District           2917      Huang High School        1910635   
         1          1  District           2949   Figueroa High School        1884411   
         2          2   Charter           1761    Shelton High School        1056600   
         3          3  District           4635  Hernandez High School        3022020   
         4          4   Charter           1468    Griffin High School         917500   
         
            Budget per Student   Maths  Reading  Maths Pass Count  Read Pass Count  \textbackslash{}
         0               655.0  223528   236810              1847             2299   
         1               639.0  226223   239335              1880             2313   
         2               600.0  146796   147441              1583             1631   
         3               652.0  358238   375131              3001             3624   
         4               625.0  122360   123043              1317             1371   
         
           Spend Group  
         0     646-675  
         1     616-645  
         2     585-615  
         3     646-675  
         4     616-645  
\end{Verbatim}
            
    \begin{Verbatim}[commandchars=\\\{\}]
{\color{incolor}In [{\color{incolor}23}]:} \PY{n}{grade\PYZus{}school\PYZus{}spend}\PY{o}{=}\PY{n}{spending\PYZus{}group}\PY{p}{(}\PY{n}{spend\PYZus{}df}\PY{p}{,}\PY{l+s+s2}{\PYZdq{}}\PY{l+s+s2}{Spend Group}\PY{l+s+s2}{\PYZdq{}}\PY{p}{)}
         \PY{n}{grade\PYZus{}school\PYZus{}spend}
\end{Verbatim}


\begin{Verbatim}[commandchars=\\\{\}]
{\color{outcolor}Out[{\color{outcolor}23}]:}   Spend Group Range  Avg Maths Score  Avg Reading Score  Maths \%  Read \%  \textbackslash{}
         0           646-675            77.05              81.01    64.06   78.37   
         1           616-645            78.06              81.43    68.96   80.95   
         2           585-615            83.53              83.84    90.53   92.47   
         3              <585            83.36              83.96    90.33   93.45   
         
            Overall \%  
         0      71.22  
         1      74.95  
         2      91.50  
         3      91.89  
\end{Verbatim}
            
    \paragraph{\texorpdfstring{The overall \% is higher for group less than
585 spend range and lowest for 646-675 spend range . Here also it can be
seen that Maths \% is greater than read \% for spend range above 616 and
lesser for other two
groups.}{The overall \% is higher for group  less than 585 spend range  and lowest for  646-675 spend range . Here also it can be seen that Maths \% is greater than read \% for spend range above 616 and lesser for other two groups.}}\label{the-overall-is-higher-for-group-less-than-585-spend-range-and-lowest-for-646-675-spend-range-.-here-also-it-can-be-seen-that-maths-is-greater-than-read-for-spend-range-above-616-and-lesser-for-other-two-groups.}

    \subsubsection{Score By School Size}\label{score-by-school-size}

    Use the spending\_student() create df for grouping by size

    \begin{Verbatim}[commandchars=\\\{\}]
{\color{incolor}In [{\color{incolor}24}]:} \PY{n}{size\PYZus{}df}\PY{o}{=}\PY{n}{spending\PYZus{}student}\PY{p}{(}\PY{p}{)}
         
         \PY{n}{size\PYZus{}df}\PY{o}{.}\PY{n}{head}\PY{p}{(}\PY{p}{)}
\end{Verbatim}


\begin{Verbatim}[commandchars=\\\{\}]
{\color{outcolor}Out[{\color{outcolor}24}]:}    School ID      Type  Student Count            School Name  School Budget  \textbackslash{}
         0          0  District           2917      Huang High School        1910635   
         1          1  District           2949   Figueroa High School        1884411   
         2          2   Charter           1761    Shelton High School        1056600   
         3          3  District           4635  Hernandez High School        3022020   
         4          4   Charter           1468    Griffin High School         917500   
         
            Budget per Student   Maths  Reading  Maths Pass Count  Read Pass Count  
         0               655.0  223528   236810              1847             2299  
         1               639.0  226223   239335              1880             2313  
         2               600.0  146796   147441              1583             1631  
         3               652.0  358238   375131              3001             3624  
         4               625.0  122360   123043              1317             1371  
\end{Verbatim}
            
    \begin{Verbatim}[commandchars=\\\{\}]
{\color{incolor}In [{\color{incolor}25}]:} \PY{c+c1}{\PYZsh{}create bins and cut based on labels}
         \PY{n}{bins\PYZus{}size}\PY{o}{=}\PY{p}{[}\PY{l+m+mi}{0}\PY{p}{,}\PY{l+m+mi}{1700}\PY{p}{,} \PY{l+m+mi}{3400}\PY{p}{,} \PY{l+m+mi}{5100}\PY{p}{]}
         \PY{n}{size\PYZus{}labels} \PY{o}{=} \PY{p}{[}\PY{l+s+s1}{\PYZsq{}}\PY{l+s+s1}{\PYZlt{}1700}\PY{l+s+s1}{\PYZsq{}}\PY{p}{,} \PY{l+s+s1}{\PYZsq{}}\PY{l+s+s1}{1700\PYZhy{}3400}\PY{l+s+s1}{\PYZsq{}}\PY{p}{,} \PY{l+s+s1}{\PYZsq{}}\PY{l+s+s1}{3400\PYZhy{}5100}\PY{l+s+s1}{\PYZsq{}}\PY{p}{]}
         \PY{n}{size\PYZus{}df}\PY{p}{[}\PY{l+s+s2}{\PYZdq{}}\PY{l+s+s2}{Size Group}\PY{l+s+s2}{\PYZdq{}}\PY{p}{]} \PY{o}{=} \PY{n}{pd}\PY{o}{.}\PY{n}{cut}\PY{p}{(}\PY{n}{spend\PYZus{}df}\PY{p}{[}\PY{l+s+s2}{\PYZdq{}}\PY{l+s+s2}{Student Count}\PY{l+s+s2}{\PYZdq{}}\PY{p}{]}\PY{p}{,}\PY{n}{bins\PYZus{}size}\PY{p}{,}\PY{n}{labels}\PY{o}{=}\PY{n}{size\PYZus{}labels} \PY{p}{)}
         \PY{n}{size\PYZus{}df}\PY{o}{.}\PY{n}{head}\PY{p}{(}\PY{p}{)}
\end{Verbatim}


\begin{Verbatim}[commandchars=\\\{\}]
{\color{outcolor}Out[{\color{outcolor}25}]:}    School ID      Type  Student Count            School Name  School Budget  \textbackslash{}
         0          0  District           2917      Huang High School        1910635   
         1          1  District           2949   Figueroa High School        1884411   
         2          2   Charter           1761    Shelton High School        1056600   
         3          3  District           4635  Hernandez High School        3022020   
         4          4   Charter           1468    Griffin High School         917500   
         
            Budget per Student   Maths  Reading  Maths Pass Count  Read Pass Count  \textbackslash{}
         0               655.0  223528   236810              1847             2299   
         1               639.0  226223   239335              1880             2313   
         2               600.0  146796   147441              1583             1631   
         3               652.0  358238   375131              3001             3624   
         4               625.0  122360   123043              1317             1371   
         
           Size Group  
         0  1700-3400  
         1  1700-3400  
         2  1700-3400  
         3  3400-5100  
         4      <1700  
\end{Verbatim}
            
    \begin{Verbatim}[commandchars=\\\{\}]
{\color{incolor}In [{\color{incolor}26}]:} \PY{c+c1}{\PYZsh{}call the spending\PYZus{}group function to calculate the  group based \PYZpc{}}
         \PY{n}{grade\PYZus{}school\PYZus{}size}\PY{o}{=}\PY{n}{spending\PYZus{}group}\PY{p}{(}\PY{n}{size\PYZus{}df}\PY{p}{,}\PY{l+s+s2}{\PYZdq{}}\PY{l+s+s2}{Size Group}\PY{l+s+s2}{\PYZdq{}}\PY{p}{)}
         \PY{c+c1}{\PYZsh{}grade\PYZus{}school\PYZus{}size.rest\PYZus{}index()}
         \PY{n}{grade\PYZus{}school\PYZus{}size}\PY{o}{.}\PY{n}{set\PYZus{}index}\PY{p}{(}\PY{l+s+s1}{\PYZsq{}}\PY{l+s+s1}{Size Group Range}\PY{l+s+s1}{\PYZsq{}}\PY{p}{)}
         \PY{n}{grade\PYZus{}school\PYZus{}size}\PY{o}{.}\PY{n}{head}\PY{p}{(}\PY{p}{)}
\end{Verbatim}


\begin{Verbatim}[commandchars=\\\{\}]
{\color{outcolor}Out[{\color{outcolor}26}]:}   Size Group Range  Avg Maths Score  Avg Reading Score  Maths \%  Read \%  \textbackslash{}
         0        1700-3400            79.89              82.40    76.51   85.37   
         1        3400-5100            77.07              80.93    64.34   78.42   
         2            <1700            83.52              83.88    90.41   92.90   
         
            Overall \%  
         0      80.94  
         1      71.38  
         2      91.66  
\end{Verbatim}
            
    The overall \% is higher for group less than 1700 size range and lowest
for 3400-5100 Size range . Here also it can be seen that Maths \% is
greater than read \% for size range all groups.

    \subsubsection{Scores by School Type}\label{scores-by-school-type}

    \begin{Verbatim}[commandchars=\\\{\}]
{\color{incolor}In [{\color{incolor}27}]:} \PY{c+c1}{\PYZsh{}Use the spending\PYZus{}student() create df for grouping by type}
         \PY{n}{type\PYZus{}df}\PY{o}{=}\PY{n}{spending\PYZus{}student}\PY{p}{(}\PY{p}{)}
         
         \PY{n}{type\PYZus{}df}\PY{o}{.}\PY{n}{head}\PY{p}{(}\PY{p}{)}
\end{Verbatim}


\begin{Verbatim}[commandchars=\\\{\}]
{\color{outcolor}Out[{\color{outcolor}27}]:}    School ID      Type  Student Count            School Name  School Budget  \textbackslash{}
         0          0  District           2917      Huang High School        1910635   
         1          1  District           2949   Figueroa High School        1884411   
         2          2   Charter           1761    Shelton High School        1056600   
         3          3  District           4635  Hernandez High School        3022020   
         4          4   Charter           1468    Griffin High School         917500   
         
            Budget per Student   Maths  Reading  Maths Pass Count  Read Pass Count  
         0               655.0  223528   236810              1847             2299  
         1               639.0  226223   239335              1880             2313  
         2               600.0  146796   147441              1583             1631  
         3               652.0  358238   375131              3001             3624  
         4               625.0  122360   123043              1317             1371  
\end{Verbatim}
            
    \begin{Verbatim}[commandchars=\\\{\}]
{\color{incolor}In [{\color{incolor}28}]:} \PY{c+c1}{\PYZsh{}call the spending\PYZus{}group function to calculate the  group based \PYZpc{}}
         \PY{n}{grade\PYZus{}school\PYZus{}district}\PY{o}{=}\PY{n}{spending\PYZus{}group}\PY{p}{(}\PY{n}{size\PYZus{}df}\PY{p}{,}\PY{l+s+s2}{\PYZdq{}}\PY{l+s+s2}{Type}\PY{l+s+s2}{\PYZdq{}}\PY{p}{)}
         \PY{n}{grade\PYZus{}school\PYZus{}district}\PY{o}{.}\PY{n}{set\PYZus{}index}\PY{p}{(}\PY{l+s+s2}{\PYZdq{}}\PY{l+s+s2}{Type Range}\PY{l+s+s2}{\PYZdq{}}\PY{p}{)}
         \PY{n}{grade\PYZus{}school\PYZus{}district}
\end{Verbatim}


\begin{Verbatim}[commandchars=\\\{\}]
{\color{outcolor}Out[{\color{outcolor}28}]:}   Type Range  Avg Maths Score  Avg Reading Score  Maths \%  Read \%  Overall \%
         0   District            76.99              80.96    64.31   78.37      71.34
         1    Charter            83.41              83.90    90.28   93.15      91.72
\end{Verbatim}
            
    The overall \% is higher for Charter type by more than 20\% .

    \subsection{Final Results:}\label{final-results}


\begin{enumerate}  
	\item It can be analayzed from above datas that Charter schools score higher pass \% than District type schools.  
	\item The schools with higher students counts has the higher pass \%
	\item For schools with highest Budget Per Student has lower Pass \% rate , which means the schools which have low    budget per students scored more marks.
	\item Also schools in top 5 are all of type Charter.
\end{enumerate}




    % Add a bibliography block to the postdoc
    
    
    
    \end{document}
